%----------------------------------------------------------------------------------------
%	PACKAGES AND THEMES
%----------------------------------------------------------------------------------------
\documentclass[aspectratio=169,xcolor=dvipsnames]{beamer}

\definecolor{links}{HTML}{2A1B81}
\hypersetup{colorlinks,linkcolor=,urlcolor=links}

\usetheme{Berkeley}

\usepackage{xcolor}
\usepackage{hyperref}
\usepackage{graphicx} % Allows including images
\usepackage{booktabs} % Allows the use of \toprule, \midrule and \bottomrule in tables

%----------------------------------------------------------------------------------------
%	TITLE PAGE
%----------------------------------------------------------------------------------------

% The title
\title[Functional MRI: Part 1 of 2]{Understanding and Interpreting \\Functional Magnetic Resonance Imaging}
\subtitle{Precision Health Boot Camp}

\author[Dr. Alexander Mark Weber] {Alexander Mark Weber}
\institute[UBC] % Your institution may be shorthand to save space
{
    % Your institution for the title page
    Department of Pediatrics, Division of Neurology \\
    University of British Columbia 
    \vskip 3pt
}
\date{9:00 - 11:00 AM \\August 9th, 2022} % Date, can be changed to a custom date


%----------------------------------------------------------------------------------------
%	PRESENTATION SLIDES
%----------------------------------------------------------------------------------------

\begin{document}

\begin{frame}
    % Print the title page as the first slide
    \titlepage
\end{frame}

\begin{frame}{Preamble}

\begin{columns}[c]
\column{0.4\textwidth}
\begin{center}
\includegraphics[width=.7\textwidth]{imgs/Lynne}
\end{center}

\includegraphics[width=.5\textwidth]{imgs/todd}%
\includegraphics[width=.5\textwidth]{imgs/tammy}
\column{0.3\textwidth}

\includegraphics[width=1\textwidth]{imgs/lit1}

\vspace{1cm}

\includegraphics[width=1\textwidth]{imgs/lit3}

\vspace{1cm}

\includegraphics[width=1\textwidth]{imgs/lit5}

\column{0.3\textwidth}

\includegraphics[width=1\textwidth]{imgs/lit2}

\vspace{1cm}

\includegraphics[width=1\textwidth]{imgs/lit4}

\end{columns}
\end{frame}

\begin{frame}{Overview}
    % Throughout your presentation, if you choose to use \section{} and \subsection{} commands, these will automatically be printed on this slide as an overview of your presentation
    \tableofcontents
\end{frame}

%------------------------------------------------
\section{Housekeeping}
%------------------------------------------------

\begin{frame}{Land Acknowledgement}

I would like to acknowledge that we are gathered today on the traditional, ancestral, and unceded territory of the Musqueam people.

\vspace{0.2cm}
\begin{center}
\includegraphics[width=.9\textwidth]{imgs/musqueam}
\end{center}

\end{frame}

%------------------------------------------------

\begin{frame}{Copyright Information}

\begin{columns}[c]
\column{0.5\textwidth}
\includegraphics[width=1\textwidth]{imgs/cc}
\column{0.5\textwidth}
\includegraphics[width=1\textwidth]{imgs/cc2}
\end{columns}

\vspace{.5cm}
Read more here: \url{https://creativecommons.org/licenses/by-sa/4.0/}

\end{frame}

%------------------------------------------------
\section{fMRI Overview}
%------------------------------------------------

\begin{frame}{fMRI Overview}
\begin{columns}[c]
\column{0.3\textwidth}
\includegraphics[width=1\textwidth]{imgs/MRI}
\tiny{https://www.northwestradiology.com/blog/page/16/}
\column{0.7\textwidth}
\begin{itemize}
\item \textbf{Magnetic:} Large magnet: 1.5 Tesla to 10 Tesla strength
\item Earth's magnetic field is 0.00005 Tesla
\item Our BCCHR Research MRI is 3T; ~ 60,000x Earth's magnetic field
\item \textbf{Resonance:} Uses radiowaves with frequencies that resonate with atomic nuclei (hydrogen, usually)
\item \textbf{Imaging:} Converts spatial frequencies and phase into images
\end{itemize}
\end{columns}
\end{frame}

%------------------------------------------------

\begin{frame}{fMRI Overview}
\begin{columns}[c]
\column<1->{0.45\textwidth}
\includegraphics[width=1\textwidth]{imgs/fmri_overview}
\tiny{Soares et al. (2016) Front Neurosci, 10(515)}

\column<2->{0.55\textwidth}
Lecture:
\begin{itemize}
\item MRI Physics
\item Brain physiology
\item fMRI acquisition
\item Task vs Resting State fMRI
\end{itemize}

Tutorial:
\begin{itemize}
\item Get data from PACS; Copy over to Sockeye; Conversion
\item BIDS
\item Preprocessing using \texttt{fmriprep}
\item Nuisance regression (two ways)
\end{itemize}
\end{columns}

\end{frame}

%------------------------------------------------
\section{Learning Objectives}
%------------------------------------------------

\begin{frame}
At the end of this session, students will be able to:
\begin{itemize}
\item Describe how MRIs produce and detect signal from the body
\item Describe how brain activity can alter fMRI signals
\item Describe the difference between task and resting state fMRI
\item How to download data from open resources online to Sockeye
\item How to run \texttt{fmriprep} on Sockeye to preprocess your fMRI data
\end{itemize}
\end{frame}


%------------------------------------------------
\section{MRI Physics}
%------------------------------------------------

\begin{frame}{MRI Physics Overview}
\begin{columns}[c]
\column{0.33\textwidth}
\includegraphics<1->[width=1\textwidth]{imgs/planet}
\column{0.33\textwidth}
\includegraphics<2->[width=1\textwidth]{imgs/gyro-0}
\column{0.33\textwidth}
\includegraphics<3->[width=1\textwidth]{imgs/protoncharge}
\end{columns}
\end{frame}

%------------------------------------------------

\begin{frame}{MRI Physics Overview}
\begin{columns}[c]
\column{0.33\textwidth}
\includegraphics<1->[width=1\textwidth]{imgs/magnet}
\column{0.33\textwidth}
\includegraphics<2->[width=1\textwidth]{imgs/extmagfield}
\column{0.33\textwidth}
\includegraphics<3->[width=1\textwidth]{imgs/precession}
\end{columns}
\end{frame}

%------------------------------------------------

\begin{frame}{MRI Physics Overview}
\begin{columns}[c]
\column{0.33\textwidth}
\includegraphics[width=1\textwidth]{imgs/magnet}
\column{0.33\textwidth}
\includegraphics[width=1\textwidth]{imgs/extmagfield}
\column{0.33\textwidth}
\includegraphics[width=1\textwidth]{imgs/precession2}
\end{columns}
\end{frame}

%------------------------------------------------

\begin{frame}{MRI Physics Overview}
\begin{columns}[c]
\column{0.6\textwidth}
\includegraphics<1->[width=1\textwidth]{imgs/cancel}
\column{0.4\textwidth}
\includegraphics<2->[width=1\textwidth]{imgs/cancel2}
\end{columns}
\end{frame}

%------------------------------------------------

\begin{frame}{MRI Physics Overview}
\begin{center}
\includegraphics[width=.6\textwidth]{imgs/rfpulse}

\includegraphics[width=.6\textwidth]{imgs/fmdial}
\end{center}
\end{frame}

%------------------------------------------------

\begin{frame}{MRI Physics Overview}
\begin{center}
\includegraphics[width=.75\textwidth]{imgs/rfpulse2}

\end{center}
\end{frame}

%------------------------------------------------

\begin{frame}{MRI Physics Overview}
\begin{columns}[c]
\column{0.5\textwidth}
\includegraphics[width=1\textwidth]{imgs/rfpulse3}
\column{0.5\textwidth}
\includegraphics[width=.7\textwidth]{imgs/faradaylaw}
\end{columns}
\end{frame}

%------------------------------------------------

\begin{frame}{MRI Physics Summary}

\begin{itemize}
\item Proton's have QM Spin
\item When placed in a large magnet (MRI) these protons precess around the main magnetic field
\item Some face up and others face down, many cancelling each other out
\item If we send a RF pulse at the same precession frequency, we can put these protons in phase: we create a transverse magnetic field, and destroy the longitudinal one
\item This transverse magnetic field can be measured thanks to Faraday's Law of Induction
\end{itemize}

\end{frame}

%------------------------------------------------

\begin{frame}{T$_{2}^{*}$ Overview}
\begin{columns}[c]
\column{0.35\textwidth}
\includegraphics[width=1\textwidth]{imgs/transrelax}
\column{0.65\textwidth}
\includegraphics[width=1\textwidth]{imgs/transrelax2}
\end{columns}

\end{frame}

%------------------------------------------------

\begin{frame}{T$_{2}^{*}$ Overview}
    \begin{block}{T2 relaxation}
        Also known as spin-spin decay: it is a time measure of the rate of decay caused by spin-spin interactions
    \end{block}
\begin{center}
\includegraphics[width=.75\textwidth]{imgs/T2curv}

\end{center}
\end{frame}

%------------------------------------------------

\begin{frame}{T$_{2}^{*}$ Overview}
\begin{columns}[c]
\column<1->{0.5\textwidth}
\includegraphics[width=1\textwidth]{imgs/t2brain}
\column<2->{0.5\textwidth}
T$_{2}$-weighted scan
\begin{itemize}
\item Water is bright
\item A mix of water/tissue is less bright (grey matter)
\item Fatty tissue is dark (white matter)
\end{itemize}
\end{columns}

\end{frame}

%------------------------------------------------

\begin{frame}{T$_{2}^{*}$ Overview}

\includegraphics[width=1\textwidth]{imgs/t2stardecay}

\tiny{Chavhan et al. DOI: 10.1148/rg.295095034}

\end{frame}

%------------------------------------------------

\begin{frame}{T$_{2}^{*}$ Overview}
\begin{columns}[c]
\column<1->{0.5\textwidth}
\includegraphics[width=1\textwidth]{imgs/deoxy}
\column<2->{0.5\textwidth}
\includegraphics[width=1\textwidth]{imgs/ferrous}
\end{columns}

\end{frame}

%------------------------------------------------
\section{BOLD Effect}
%------------------------------------------------

\begin{frame}{Neurovascular Coupling}

\includegraphics[width=.7\textwidth]{imgs/brainveins}

\tiny{Zlokovic, B. and Apuzzo, M. (1998). Strategies to Circumvent Vascular Barriers of the Central Nervous System. Neurosurgery 43, 877-878.}
\end{frame}

%------------------------------------------------

\begin{frame}{Neurovascular Coupling}

\includegraphics[width=.75\textwidth]{imgs/neurovasccoupling}

\end{frame}

%------------------------------------------------

\begin{frame}{Neurovascular Coupling}

\includegraphics[width=.5\textwidth]{imgs/neurovasc3}

\tiny{D’Esposito et al. (2003). Nature Reviews Neuroscience, 4}
\end{frame}

%------------------------------------------------

\begin{frame}{BOLD Effect}
\begin{columns}[c]
\column<1->{0.5\textwidth}
\includegraphics[width=1\textwidth]{imgs/neurovasccoupling2}
\column<2->{0.5\textwidth}
\includegraphics[width=1\textwidth]{imgs/bold}
\end{columns}

\tiny{Harris et al.  (2011). Developmental Cognitive Neuroscience, 1, 3}
\end{frame}

%------------------------------------------------

\begin{frame}{BOLD Effect}
\begin{columns}[c]
\column<1->{0.5\textwidth}
\includegraphics[width=1\textwidth]{imgs/BOLDeffect}
\tiny{Dogil et al. (2002). Journal of Neurolinguistics, 15(1), 59-90}
\column<1->{0.5\textwidth}
\begin{itemize}
\item<1-> Under normal conditions, \textcolor{red}{oxygenated hemoglobin} is converted to \textcolor{blue}{deoxygenated hemoglobin} within the capillary bed at a constant rate
\item<2-> When neurons become \textbf{active}, the vascular system supplies more \textcolor{red}{oxygenated hemoglobin} than is needed via an overcompensatory increase in blood flow.
\item<3-> The result is a net \textbf{decrease} in \textcolor{red}{deoxygenated hemoglobin} and a corresponding \textbf{decrease in signal loss} due to T$_{2}^{*}$ effects
\end{itemize}
\end{columns}
\end{frame}

%------------------------------------------------

\begin{frame}{Hemodynamic Response Function}

\includegraphics[width=1\textwidth]{imgs/HRF}
\tiny{Sigita Cinciute (2019) PeerJ, Mar 25;7e6621}

\normalsize{[Note: Please see work by Todd Woodward and others about how this shape should \textbf{NOT} be assumed]}
\end{frame}

%------------------------------------------------
\section{fMRI Acquisition}
%------------------------------------------------

\begin{frame}{fMRI Acquisition}
\includegraphics[width=.8\textwidth]{imgs/epipulsesignal}

\begin{center}
\includegraphics[width=.25\textwidth]{imgs/epikspace}
\end{center}

\tiny{https://mriquestions.com/echo-planar-imaging.html}
\end{frame}

%------------------------------------------------

\begin{frame}{fMRI Acquisition}
\includegraphics[width=1\textwidth]{imgs/fmrislices}
\end{frame}

%------------------------------------------------

\begin{frame}{fMRI Acquisition}
\includegraphics[width=1\textwidth]{imgs/fmritimeseries}
\end{frame}

%------------------------------------------------

\begin{frame}{fMRI Acquisition: Task-based}

\includegraphics[width=.7\textwidth]{imgs/taskfmri}

\tiny{Gary Glover. (2011). Neurosurg Clin N Am, 22(2): 133-139}
\end{frame}

%------------------------------------------------

\begin{frame}{fMRI Acquisition: Resting state}
\includegraphics[width=.7\textwidth]{imgs/restingfmri}

\tiny{Fox and Greicius. (2010). Front Syst Neurosci, 4(19)}
\end{frame}

%------------------------------------------------
\section{fMRI Analysis}
%------------------------------------------------

\begin{frame}{fMRI Analysis: Preprocessing}
\begin{columns}[c]
\column<1->{0.7\textwidth}
\includegraphics[width=1\textwidth]{imgs/preprocess}

\tiny{Markiewicz et al. (2018). OHBM Poster \#2305}

\column<2->{0.3\textwidth}
\begin{itemize}
\item Brain extraction
\item Motion correction
\item Slice-timing correction
\item Susceptibility distortion correction
\item Registration and Normalization
\item Estimate noise/confounds
\item Data quality check
\end{itemize}
\end{columns}
\end{frame}

%------------------------------------------------
%
%\begin{frame}{fMRI Analysis: Preprocessing}
%
%\url{https://openneuro.org/}
%
%\vspace{0.2cm}
%\includegraphics[width=1\textwidth]{imgs/openneuro}
%
%\end{frame}

%------------------------------------------------

\section{Sockeye Tutorial and fmriprep}

\begin{frame}{Sockeye}

\end{frame}

%----------------------------------------------------------------------------------------

\end{document}